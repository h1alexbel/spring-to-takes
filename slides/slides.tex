% (The MIT License)
%
% Copyright (c) 2023-2024 Yegor Bugayenko
%
% Permission is hereby granted, free of charge, to any person obtaining a copy
% of this software and associated documentation files (the 'Software'), to deal
% in the Software without restriction, including without limitation the rights
% to use, copy, modify, merge, publish, distribute, sublicense, and/or sell
% copies of the Software, and to permit persons to whom the Software is
% furnished to do so, subject to the following conditions:
%
% The above copyright notice and this permission notice shall be included in all
% copies or substantial portions of the Software.
%
% THE SOFTWARE IS PROVIDED 'AS IS', WITHOUT WARRANTY OF ANY KIND, EXPRESS OR
% IMPLIED, INCLUDING BUT NOT LIMITED TO THE WARRANTIES OF MERCHANTABILITY,
% FITNESS FOR A PARTICULAR PURPOSE AND NONINFRINGEMENT. IN NO EVENT SHALL THE
% AUTHORS OR COPYRIGHT HOLDERS BE LIABLE FOR ANY CLAIM, DAMAGES OR OTHER
% LIABILITY, WHETHER IN AN ACTION OF CONTRACT, TORT OR OTHERWISE, ARISING FROM,
% OUT OF OR IN CONNECTION WITH THE SOFTWARE OR THE USE OR OTHER DEALINGS IN THE
% SOFTWARE.

\documentclass{article}
\usepackage{../sqm}
\usepackage{fontawesome5}
\newcommand*\thetitle{Spring and Takes}
\newcommand*\thesubtitle{Layer Complexity}
\begin{document}

%  spring model-dao-service-controller example
%  problems
%  takes soft layers + diagonal object scaling
%  missed: spring ecosystem vs. takes ecosystem

\plush{\sqmTitlePage{15}}

\plush{%
  \pptSection{About us}
  \begin{multicols}{2}
    \pptBanner[green] {Aliaksei Bialiauski}
    \begin{itemize}
      \item[] Integration/Java Developer at \href{https://www.solvd.com/}{Solvd}.
      \item[] \faGithub \hspace{0.1cm} \href{https://github.com/h1alexbel}{@h1alexbel}
      \item[] \faTelegram \hspace{0.1cm} \href{https://t.me/h1alexbel}{@h1alexbel}
    \end{itemize}
    \par\columnbreak
    \pptBanner[green] {Ivan Ivanchuk}
    \begin{itemize}
      \item[] Kotlin Developer at \href{https://www.vtb.ru/}{Samokat}.
      \item[] \faGithub \hspace{0.1cm}  \href{https://github.com/l3r8yJ}{@l3r8yJ}
      \item[] \faTelegram \hspace{0.1cm} \href{https://t.me/l3r8y}{@l3r8y}
    \end{itemize}
  \end{multicols}
}

\pptBanner{Spring}
\begin{multicols}{2}
{\small\begin{ffcode}
@Data
class Invoice {
  private int id;
  private String label;
}
\end{ffcode}
}
\par\columnbreak\par
{\small\begin{ffcode}
interface InvoiceRepository extends Repository<Invoice> {}
\end{ffcode}
}
\end{multicols}
\plush{}

\begin{multicols}{2}

{\small\begin{ffcode}
class InvoiceService {}
\end{ffcode}
}
\par\columnbreak\par
{\small\begin{ffcode}
@RestController
class InvoiceController {
  private final InvoiceSvc svc;
  private final Mapper map;
  @PostMapping
  void create(InvoiceDto dto) {
    this.svc.create(this.map.toData(dto));
  }
}
\end{ffcode}
}
\end{multicols}
\plush{}

\plush{%
  \pptBanner{Problems}
}

\plush{%
  \pptBanner{Handling complexity in Takes}
}

\plush{%
    \pptSection{Summary}
    \begin{itemize}
        \item \ldots
    \end{itemize}
}

\pitch{\pptQuote{adam-bender.jpg}{Code coverage \emph{does not guarantee} that the covered lines or branches have been tested correctly, it just guarantees that they have been executed by a test. But a low code coverage number \emph{does guarantee} that large areas of the product are going completely untested by automation on every single deployment.}{\textit{Code Coverage Best Practices}, Carlos Arguelles, Marko Ivankovi{\'c}, \emph{Adam Bender}, \href{https://testing.googleblog.com/2020/08/code-coverage-best-practices.html}{Google Blog}, 2020}}

\plush{%
    \pptSection{Q\&A} \par
    \begin{itemize}
        \item[] \href {https://h1alexbel.github.io/}{h1alexbel.github.io}
        \item[] \href {https://www.l3r8y.ru/} {www.l3r8y.ru}
    \end{itemize}
}

%\plush{}
%
%\pptBanner{Example, Part II}
%\begin{multicols}{2}
%Live Code:\par
%{\small\begin{ffcode}
%(*@\textcolor{green}{int fibonacci(int n) \{} @*)
%  (*@\textcolor{green}{if (n <= 0) \{} @*)
%    return 0;
%  (*@\textcolor{green}{\}} @*)
%  (*@\textcolor{green}{if (n <= 2) \{} @*)
%    (*@\textcolor{green}{return 1;} @*)
%  (*@\textcolor{green}{\}} @*)
%  (*@\textcolor{green}{return fibonacci(n-1)}@*)
%    (*@\textcolor{green}{+ fibonacci(n-2);}@*)
%(*@\textcolor{green}{\}}@*)
%\end{ffcode}
%}
%\par\columnbreak\par
%Test Code:\par
%{\small\begin{ffcode}
%assert fibonacci(1) == 1;
%assert fibonacci(2) == 1;
%
%assert fibonacci(9) == 34;
%assert fibonacci(10) == 55;
%\end{ffcode}
%}
%\( C = 9/10 = 90\% \)
%\end{multicols}
%\plush{}

\end{document}
